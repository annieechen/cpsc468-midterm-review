\documentclass[10pt]{article}
\usepackage{geometry}
\usepackage{amsmath}
\usepackage{amssymb}
\usepackage{amsthm}
\usepackage{algorithmicx}
\usepackage{algorithm}
\usepackage{algpseudocode}
\usepackage{graphicx}
\usepackage{float}
\usepackage{ulem}
\usepackage{hyperref}
\hypersetup{
  colorlinks=true,
  linkcolor=blue,
}
% \usepackage{listings,lstautogobble}
\geometry{margin=0.75in}
\setlength\parindent{0pt}

\title{CPSC 468 PSET Midterm Review}
\author{Anton Xue [ayx2]}
\date{}

\mathchardef\mhyphen="2D

% \theoremstyle{definition}
\newtheorem{definition}{Definition}[section]
\newtheorem{theorem}{Theorem}[section]
\newtheorem{claim}{Claim}[section]
\newtheorem{example}{Example}[section]

\begin{document}
\maketitle

\newcommand{\E}{\mathrm{E}}
\newcommand{\Var}{\mathrm{Var}}
\newcommand{\Real}{\mathrm{Re}}
\newcommand{\Imag}{\mathrm{Im}}
\newcommand{\res}{\mathrm{res}}
\newcommand{\NP}{\mathrm{NP}}
\newcommand{\NPC}{\mathrm{NP\mhyphen Complete}}
\newcommand{\coNP}{\mathrm{coNP}}
\newcommand{\coNPC}{\mathrm{coNP\mhyphen Complete}}
\newcommand{\DP}{\mathrm{DP}}
\newcommand{\PTIME}{\mathrm{P}}
\newcommand{\AL}{\mathrm{AL}}
\newcommand{\INDSET}{\mathrm{INDSET}}
\newcommand{\NOTINDSET}{\overline{\mathrm{INDSET}}}
\newcommand{\EXACTINDSET}{\mathrm{EXACT\mhyphen INDSET}}
\newcommand{\SAT}{\mathrm{SAT}}
\newcommand{\UNSAT}{\overline{\mathrm{SAT}}}
\newcommand{\ACCEPT}{\mathrm{ACCEPT}}
\newcommand{\REJECT}{\mathrm{REJECT}}

\subsection*{Chapter 1}
\begin{definition}[Turing Machine]
  A $k$-tape Turing Machine $M$ is described by a tuple $(\Gamma, Q, \delta)$.
  Assume $k \geq 2$, with $1$ read-only input tape and $k - 1$ work tapes.
  The last work tape is assumed to be the output tape.
  \begin{itemize}
    \item
      $\Gamma$ the finite alphabet of symbols that $M$ may have on its tapes.
      Assume that $\Gamma$ contains at least $\{0,1,\square,\triangleright\}$.

    \item
      $Q$ a finite set of possible states $M$'s state register may be in.
      Assume that $Q$ contains a $q_{start}$ and $q_{halt}$.

    \item
      $\delta : Q \times \Gamma^k \to Q
      \times \Gamma^{k - 1} \times \{L, S, R\}^k$ a transition function for
      $M$ that takes in the current state and each head's read, and outputs
      the next state, with $k - 1$ writes on all the work tapes, and movement
      direction for all $k$ tapes.
  \end{itemize}
\end{definition}

\begin{definition}[Computing a Function]
  Let $f : \{0, 1\}^\ast \to \{0, 1\}^\ast$ and
  $T : \mathbb{N} \to \mathbb{N}$, with $M$ a TM.
  We say that $M$ computes $f$ if for every $x \in \{0, 1\}^\ast$, if
  $M$ is initialized to the start configuration on input $x$, then it halts
  with $f(x)$ on the output tape.
  We say $M$ computes $f$ in $T(n)$-time if its computation on every
  $x$ requires at most $T(|x|)$ steps.
\end{definition}

\begin{definition}[Time Constructible]
  A function $T: \mathbb{N} \to \mathbb{N}$ is time constructible if
  $T(n) \geq n$ and there is a TM $M$ that computes the function
  $x \mapsto \llcorner T(|x|) \lrcorner$ in time $T(n)$.
  $T(n) \geq n$ is to allow the algorithm to read its input.
\end{definition}

\begin{example}[Time Constructible Functions]
  Some time constructible functions are $n$, $n \log n$, $n^2$ and $2^n$.
\end{example}

\begin{claim}
  For every $f : \{0, 1\}^\ast \to \{0, 1\}$ and time constructible
  $T : \mathbb{N} \to \mathbb{N}$, if $f$ is computable in time $T(n)$ by
  some TM $M$ using alphabet $\Gamma$, then it is able to compute the same
  function using $\{0, 1, \square, \triangleright\}$ in 
  $\left(c \log_2 | \Gamma| \right) \cdot T(n)$.
  This is because we may express each symbol of $\Gamma$ using
  $\log |\Gamma|$ binary bits, with some constant $c$ overhead.
\end{claim}

\begin{claim}
  A $k$-tape TM can have its $k - 1$ work tapes simuliated by a single tape by
  interleaving the $k$ tapes together.
\end{claim}

\begin{definition}[Oblivious Turing Machine]
  An oblivious TM's head movement depends on the length of the input, not
  the contents of the input.
  Every TM can be simulated by an oblivious TM.
\end{definition}

\begin{claim}[Turing Machines as Strings]
  Every binary string $x \in \{0, 1\}^\ast$ represents some TM, and every TM
  is represented by infinite such strings (think: comments in a language).
  The machine represented by $x$ is denoted $M_x$.
\end{claim}

\begin{claim}[Universal Turing Machine]
  There exists a TM $\mathcal{U}$ such that for every
  $x, \alpha \in \{0, 1\}^\ast$, $\mathcal{U}(x, a) = M_\alpha (x)$, where
  $M_\alpha$ denotes the TM represented by $\alpha$.
  Moreover, if $M_\alpha$ halts on input $x$ within $T$ steps, then
  $\mathcal{U}_\alpha (x)$ halts within $C T \log T$ steps, where $C$ is
  a number independent of $|x|$, and depends only on $M_\alpha$'s alphabet
  size, number of tapes, and number of states.
  In other words, the cost of simulating any machine $M_\alpha$ has a
  logarithmic overhead.
\end{claim}

\subsection*{Chapter 2}



\subsection*{Chapter 3}



\subsection*{Chapter 4}


\subsection*{Chapter 5}



\subsection*{Chapter 6}




\end{document}

