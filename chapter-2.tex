\subsection*{Chapter 2: $\NP$ and $\NP$-completeness}
\begin{note}[Non-deterministic Turing Machine]
  A non-deterministic TM is endowed with two transition functions
  $\delta_0$ and $\delta_1$ along with a special accept state $q_{accept}$.
  The $\NDTM$ may use either transition function per time step.
  For every input $x$, we say that $M(x) = 1$ if there \textbf{exists}
  some sequence of transition function choises
  that would cause $M$ to reach $q_{accept}$.
  Otherwise, if every sequence of non-deterministic choices causes $M$ to halt
  on $x$ without reaching $q_{accept}$, then we say that $M(x) = 0$.
  $M$ runs in time $T(n)$ if for every input $x \in \{0, 1\}^\ast$ and every
  sequence of non-deterministic choices, $M$ reaches eitehr the halting state
  or $q_{accept}$ within $T(|x|)$ steps.
\end{note}

\begin{note}[$\NTIME$]
  For every function $T : \mathbb{N} \to \mathbb{N}$ and
  $L \subseteq \{0, 1\}^\ast$, we say that $L \in \NTIME(T(n))$ if there is
  a constant $c > 0$ and a $c \cdot T(n)$-time NDTM $M$ such that for
  every $x \in \{0, 1\}^\ast$, we have $x \in L \iff M(x) = 1$.
\end{note}

\begin{note}[$\NP$]
  $\NP = \bigcup_{c \in \mathbb{N}} \NTIME(n^c)$
\end{note}

\begin{note}[]
  A language $L \subseteq \{0, 1\}^\ast$ is in $\NP$ if there exists a
  polynomial $p : \mathbb{N} \to \mathbb{N}$ and a polynomial time TM $M$
  (called the \textbf{verifier} for $L$) $\st$ for every
  $x \in \{0, 1\}^\ast$, we have
  $x \in L \iff \exists u \in \{0, 1\}^{p(|x|)} \st M(x, u) = 1$.
  If $x \in L$ and $u \in \{0, 1\}^{p(|x|)}$ satisfy $M(x, u) = 1$,
  we call $u$ a \textbf{certificate} for $x$ (with respect to $L$ and $M$).
  $\NP$ is the class of languages for which we can tell if
  $u$ is a solution to the problem $x \in L$ in polynomial time.
\end{note}

\begin{note}[Reductions, $\NP$-hardness, and $\NP$-completeness]
  We say that a language $L \subseteq \{0, 1\}^\ast$ is a polynomial time
  \textbf{Karp reducible} to a language $L^\prime \subseteq \{0, 1\}^\ast$ if
  there is a polynomial time computable function
  $f : \{0, 1\}^\ast \to \{0, 1\}^\ast$ such that for
  every $x \in \{0, 1\}^\ast$, we have $x \in L \iff f(x) \in L^\prime$.
  We say that $L^\prime$ is $\NP$-hard if $L \leq_p L^\prime$ for every
  $L \in \NP$.
  We say that $L^\prime$ is $\NP$-complete if $L^\prime$ is $\NP$-hard and
  $L^\prime \in \NP$.
\end{note}

\begin{note}[Cook Reduction]
  A reduction computed by a deterministic polynomial time oracle TM.
\end{note}

\begin{note}[Transitivity]
  If $L \leq_p L^\prime$, and $L^\prime \leq_p L^{\prime \prime}$, then
  $L \leq_p L^{\prime \prime}$.
\end{note}

\begin{note}
  If $L$ is $\NP$-hard and $L \in \PTIME$, then $\PTIME = \NP$.
\end{note}

\begin{note}
  If $L$ is $\NP$ complete then $L \in \PTIME \iff \PTIME = \NP$.
\end{note}

\begin{note}[TMSAT]
  $\TMSAT = \{ (\alpha, x, 1^n, 1^t) :
    \text{$\exists u \in \{0, 1\}^n \st
          M_a (x, u) = 1$ within $t$ steps} \}$
\end{note}

\begin{note}[Decision vs Search]
  Suppose that $\PTIME = \NP$, then for every $\NP$ language $L$ there exists
  a polynomial time TM $B$ such that on input $x \in L$ outputs a certificate
  for $x$.
  That is, $x \in L \iff \exists u \in \{ 0, 1\}^{p(|x|)} \st M(x, u) = 1$
  where $p$ is some polynomial and $M$ is a polynomial time TM, then on
  input $x \in L$, we have $B(x)$ as the string $u \in \{0, 1\}^{p(|x|)}$
  satisfying $M(x, B(x)) = 1$.
\end{note}

\begin{note}[Complement Language]
  If $L \subseteq \{0, 1\}^\ast$, then we denote $\overline{L}$ to be the
  complement of $L$.
  That is, $\overline{L} = \{0, 1\}^\ast \setminus L$.
\end{note}

\begin{note}[$\coNP$]
  $\coNP = \{L : \overline{L} \in \NP\}$
\end{note}

\begin{note}[$\coNP$ Alternative Definition]
  For every $L \subseteq \{0, 1\}^\ast$, we say $L \in \coNP$ if there
  is a polynomial $p : \mathbb{N} \to \mathbb{N}$ and a polynomial time
  TM $M$ $\st$ for every $x \in \{0, 1\}^\ast$, we have
  $x \in L \iff \forall u \in \{0, 1\}^{p(|x|)}$ with $M(x, u) = 1$.
\end{note}

\begin{note}
  $\coNP$ is the class of problems for which we can reject proposed solutions
  in polynomial time.
\end{note}

\begin{note}
  $\PTIME \subseteq \NP \cap \coNP$
\end{note}

\begin{note}[$\EXP$]
  $\EXP = \bigcup_{c \geq 1} \DTIME\left(2^{n^c}\right)$
\end{note}

\begin{note}[$\NEXP$]
  $\NEXP = \bigcup_{c \geq 1} \NTIME\left(2^{n^c}\right)$
\end{note}
