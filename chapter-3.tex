\subsection*{Chapter 3: Diagonalization}
\begin{note}[Main Idea]
  For two complexity classes $C_1$ and $C_2$, we show
  $C_1 \subsetneq C_2$ by presenting $L \st L \in C_2$ but $L \not\in C_1$.
\end{note}

\begin{note}[Time Hierarchy Theorem]
  If $f, g$ are time constructible function satisfying
  $f(n) \log f(n) = o (g(n))$, then we have
  $\DTIME(f(n)) \subsetneq \DTIME(g(n))$.
\end{note}

\begin{note}[Sketch for $\DTIME(n) \subsetneq \DTIME\left(n^{1.5}\right)$]
  For the ``diagonal'' machine $D$:
  on input $x$ run for $|x|^{1.4}$ steps on the universal TM $\mathcal{U}$
  to simulate $M_x (x)$.
  If $\mathcal{U}$ outputs a bit $b \in \{0, 1\}$, we yield the opposite,
  which is $1 - b$, and $0$ if it fails to halt in time $|x|^{1.4}$.
  By definition, $D$ always halts within $n^{1.4}$ steps and so any language
  decided by $D$ is in $\DTIME\left(n^{1.5}\right)$.
  There is no machine $M$ that always halts within $n$ steps that can emulate
  $D$ because $D$ will have captured its result and yielded the opposite.
  Thus, $L \in \DTIME\left(n^{1.5}\right)$, but $L \not\in \DTIME(n)$.
\end{note}

\begin{note}[Non-deterministic Time Hierarchy Theorem]
  If $f, g$ are time constructible funcitons satisfying the bound
  $f(n + 1) = o(g(n))$,
  then $\NTIME(f(n)) \subsetneq \NTIME(g(n))$.
\end{note}

\begin{note}[Sketch for $\NTIME(n) \subsetneq \NTIME\left(n^{1.5}\right)$]
Apply lazy diagonalization.
Define $f : \mathbb{N} \to \mathbb{N}$ as follows: $f(1) = 2$
and $f(i + 1) = 2^{f(i)^{1.2}}$.
We will find some $n$ such that $f(i) < n \leq f(i + 1)$.
The goal for our diagonal machine is to flip the result on the
set $\{ 1^n : f(i) < n \leq f(i + 1)\}$.
Define $D$ as:
If (1) $f(i) < n < f(i + 1)$ then simulate $M_i$ on input $1 ^{n + 1}$ using
non-determinism in $n^{1.1}$ time and output.
Otherwise if (2) $n = f(i + 1)$ accept $1^n$ iff $M_i$ rejects
$1^{f(i) + 1}$ in $(f(i) + 1)^{1.1}$ time.
Thus, if $f(i) < n < f(i + 1)$ then
$D\left(1^n\right) = M_i \left(1^{n + 1}\right)$, otherwise
$D\left(1^{f(i + 1)}\right) \neq M_i \left(1^{f(i) + 1}\right)$.
\end{note}

\begin{note}[Ladner's Theorem]
  Suppose that $\PTIME \neq \NP$, then there exists a language
  $L \in \NP \setminus \PTIME$ that is not $\NP$-complete.
\end{note}

\begin{note}[Sketch for Ladner's Theorem]
  We define $A$ as follows:
  \begin{align*}
    A = \{ x : x \in \SAT, f(|x|) \mod 2 = 0\}
  \end{align*}
  The goal is to make $f : \mathbb{N} \to \mathbb{N}$
  confusing as shit to compute to fool anything in $P$
  trying to decide $A$, and fuck up anything in $\NP$ trying to reduce to $A$.
  We want $f$ to be unbounded, and monotonic increasing.
  This ensures that -- by construction, if $f$ plateaus at an even number then
  $\SAT \in \PTIME$, while if $f$ pleateaus at an odd number, then it is
  $\NP$-complete.
  Let $f(0) = f(1) = 2$, and $M_f$ the TM that computes $f$.
  The computation of $f$ proceeds in two stages:
  \begin{enumerate}
    \item [(1)]
        Let $|x| = n$, and for $n$ time steps use $M_f$ to compute
        $f(0), f(1), \dots$ until we run out of time at some
        $f(y) = k$.

    \item [(2)]
      If $k = 2i$ is even, then we try to fuck with $\PTIME$.
      We show that if the $i$th TM were to somehow decide $A$, then this would
      lead to contradictions.
      Given $|x| = n$ time, we enumerate the strings $z \in \{0, 1\}^\ast$ in
      some lexicographical order.
      Using a decider for $\SAT$, we check if $z \in \SAT$, and if $M_i (z)$
      accepts $z$ as input.
      If during our $n$ time steps of runtime, we \textbf{do not} find any
      conflicts, then we return an even number $f(|x|) = k$ to allow more time.
      Otherwise we set $f(|x|) = k + 1$.

      If $k = 2i - 1$ is odd, we show the $\SAT$ reductions some business.
      Let $F_1, F_2, \dots$ be an enumeration of TMs that reduce $\SAT$ into
      $A$.
      We give these things a total of $|x| = n$ time to work, or maybe even
      each of them $n$ time, who cares?
      If any of them \textbf{do output} something,
      then we yield $f(|x|) = k$ odd, and otherwise $f(|x|) = k + 1$ even.
  \end{enumerate}

  Assume by contradiction that $A \in \PTIME$.
  This implies that there is some $M_i$ that tryna get $A$ in polynomial time.
  Hence some value of $f$ would have gotten to some $M_i$, iterated at
  the appropriate $z$, and then started having consistent values for which the
  stuff do not disagree.
  This would cause to plateau at an even value, which also implies that
  $\SAT \in \PTIME$.

  If $A$ is $\NP$-complete, then we see that there is some reduction TM $F_i$
  that our function $f$ will have reached at some point.
  The existence of this $F_i$ will cause $f$ to plateau at an odd value.
  However, this implies that because $f$ is even for finite input, that $A$
  is a finite language, and hence not $\NP$-complete because finite languages
  are in $\PTIME$.

\end{note}

