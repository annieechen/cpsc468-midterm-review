\subsection*{Chapter 4: Space Complexity}
\begin{note}[Space-bounded Computation]
  Let $S : \mathbb{N} \to \mathbb{N}$ and $L \subseteq \{0, 1\}^\ast$.
  We say that $L \in \SPACE(s(n))$ if there is a constant $c$ and
  a deterministic TM $M$ that
  decides $L$ with at most $c \cdot s(n)$ locations on
  $M$'s work tapes (excluding the input tape) are ever visited by $M$'s head
  during its computation on every input of length $n$.
  Likewise, say that $L \in \NSPACE(s(n))$ if there is an $\NDTM$ $M$ deciding
  $L$ that never uses more than $c \cdot s(n)$ non-blank tape locations on
  length $n$ inputs, regardless of its non-deterministic choices.
\end{note}

\begin{note}
  $\DTIME(S(n)) \subseteq \SPACE(S(n))$ since a TM can access only one
  tape cell per step, so a $\SPACE(S(n))$ TM can run for much longer than
  $S(n)$ steps.
  In fact, halting behavior can run for as much as $2^{\Omega (S(n))}$ steps
  -- as an example, a counter machine that counts from $1$ to
  $2^{S(n) - 1}$.
  Any language that is in $\SPACE(S(n))$ is
  in $\DTIME\left(2^{O(S(n))}\right)$.
\end{note}

\begin{note}
  For space constructible $S: \mathbb{N} \to \mathbb{N}$:
  $\DTIME(S(n)) \subseteq \SPACE(S(n)) \subseteq \NSPACE(S(n))
  \subseteq \DTIME\left(2^{O(S(n))}\right)$.
\end{note}

\begin{note}[Configuration Graph]
  A configuration of a TM $M$ consists of the contents of all non-blank
  entries on $M$'s tapes, along with its state and head position at a
  particular point in execution.
  For eveyr space $S(n)$ TM $M$ and input $x \in \{0, 1\}^\ast$, the
  configuration graph of $M$ on nput $x$, denote $G_{M, x}$ is a directed
  graph whose nodes correspond to all possible configurations of $M$, where
  the input contains the value $x$ and the work tapes have at most $S(|x|)$
  non-blank cells.
  The graph has a directed edge from a configuration $C$ to a configuration
  $C^\prime$ if $C^\prime$ can be reached from $C$ in one step according
  to $M$'s transition function.
  If $M$ is deterministic, then the graph has out-degree one, and if $M$
  is non-deterministic, then the grpah has outdegree at most two.
  Let $G_{M, x}$ be the configuration graph of a space-$S(n)$ machine $M$ on
  some input $x \st |x| = n$:
  \begin{itemize}
    \item
      Every vertex in $G_{M, x}$ can be described using $cS(n)$ bits for some
      constant $c$ (depending on $M$'s alphabet size and number of tapes) and
      in particular, $G_{M, x}$ has at most $2^{cS(n)}$ vertices.

    \item
      There is an $O(S(n))$-size CNF formula $\varphi_{M, x}$ such that
      for every two strings $C$ and $C^\prime$, we have
      $\varphi_{M, x} (C, C^\prime) = 1$ iff $C$ and $C^\prime$ encode two
      neighboring configurations in $G_{M, x}$.
      This can be expressed as an AND of many checks to see if the transition
      of the machine state is valid.
  \end{itemize}
\end{note}

\begin{note}[Space Complexity Classes]
  We can think of $\PSPACE$ and $\NPSPACE$ as analogs to time
  classes $\PTIME$ and $\NP$.

  \begin{itemize}
    \item
      $\PSPACE = \bigcup_{c \geq 1} \SPACE(n^c)$

    \item
      $\NPSPACE = \bigcup_{c \geq 1} \NSPACE(n^c)$

    \item
      $\LSPACE = \SPACE(\log n)$

    \item
      $\NLSPACE = \NSPACE(\log n)$
  \end{itemize}
\end{note}

\begin{note}[Space Hierarchy Theorem]
  If $f, g$ are space constructible with
  $f(n) = o(g(n)): \SPACE(f(n)) \subsetneq \SPACE(g(n))$.
\end{note}

\begin{note}
  We do not know if $\PTIME = \PSPACE$, although we think no.
  Note that $\NP \subseteq \PSPACE$.
\end{note}

\begin{note}[$\PSPACE$-completeness]
  A language $L^\prime$ is $\PSPACE$-hard if for every $L \in \PSPACE$,
  $L \leq_p L^\prime$.
  If in addition $L^\prime \in \PSPACE$, then $L^\prime$ is
  $\PSPACE$-complete.
\end{note}

\begin{note}[$\SPACE \TMSAT$]
  $\SPACE \TMSAT = \left\{ (M, w, 1^n) :
    \text{$\DTM M$ accepts $w$ in space $n$} \right\}$.
\end{note}

\begin{note}[Quantified Boolean Formula]
  A $\QBF$ is of form
  $Q_1 x_1 . Q_2 x_2 . \dots . Q_n x_n . \varphi (x_1, x_2, \dots, x_n)$
  where each $Q_i \in \{\exists, \forall \}$, $x_i$ range over $\{0, 1\}$,
  and $\varphi$ is a plain, unquantified formula.
\end{note}

\begin{note}
  Since all variables of a $\QBF$ are bounded by some quantifier, $\QBF$ is
  always either $1$ or $0$ as a result.
\end{note}

\begin{note}($\TQBF$)
  $\TQBF$ is $\PSPACE$ complete.
\end{note}

\begin{note}[Sketch for $\TQBF \in \PSPACE$]
  Let $n$ be the number of quantified variables in $\TQBF$ $\Psi$,
  and proceed by induction.
  If $n = 0$, then we are done.
  Otherwise for $n > 0$, the formula will have form
  $Q_n x_n . Q_{n - 1} x_{n - 1} . \dots . Q_1 x_1 . \varphi$.
  If $Q$ is $\exists$, we check if either $\Psi |_{x_n = 1}$ or
  $\Psi_{x_n = 0}$ is true.
  If $Q$ is $\forall$, we check if both $\Psi |_{x_n = 1}$ and
  $\Psi_{x_n = 1}$ are true.
\end{note}

\begin{note}[Sketch for $\TQBF$ is $\PSPACE$-hard]
  For some $M$ that can decide $L$ in $S(n)$ space, we can encode the
  configuration graph as a $\TQBF$ $\Psi$
  such that on input $x \in \{0, 1\}^n$, $\Psi \in \TQBF \iff M(x) = 1$.
  Observe that because $M$ can decide $L$ in $S(n)$ space, we thus need
  $S(n)$ bits of information to encode $M$'s configuration graph on input $x$
  We use the property that every configuration graph for $M$ on input $x$ has
  a formula $\varphi_{M, x}$ such that for two strings,
  $C, C^\prime \in \{0, 1\}^m$, we have $\varphi_{M} (C, C^\prime) = 1$
  iff $C$ and $C^\prime$ encode two adjacent configurations in the
  configuration graph.
  When we plug in $C_{start}$ and $C_{accept}$, this gets us what we want.
  Define $\Psi$ by induction.

  \textbf{Naive solution}: let $\Psi_i (C, C^\prime) = 1$ iff there is a
  path of length at most $2^i$ from $C$ to $C^\prime$ in $G_{M, x}$.
  Define the formula
  $\Psi_{i} (C, C^\prime) = \exists C^{\prime \prime}
    \Psi_{i - 1} (C, C^{\prime \prime}) \land
    \psi_{i - 1} (C^{\prime \prime}, C^\prime)$.
  However this is a problem because $\Psi_i$ may double each layer.

  \textbf{Less naive}:
  $\Psi_{i} (C,C^\prime)=\exists C^{\prime\prime} . \forall D_1 . \forall D_2.
  \left( (D_1 = C \land D_2 = C^{\prime \prime}) \lor
         (D_1 = C^{\prime \prime} \land D_2 = C^\prime) \right) \implies
    \psi_{i - 1} (D_1, D_2)$.
\end{note}


