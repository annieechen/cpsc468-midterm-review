\subsection*{Chapter 5: The Polynomial Hierarchy and Alterations}
\begin{note}[The Class $\Sigma_{2}^{p}$]
The class $\Sigma_{2}^{p}$ is the set of all languages for which
there exists a polynomial TM $M$
and a polynomial $q$ such that
$x \in L \iff \exists u \in \{0,1\}^{q(|x|)} \forall v \in
\{0,1\}^{q(|x|)} \st M(x, u, v) = 1 $

Note that $\Sigma_{2}^{p}$ contains the
classes $\textbf{NP}$ and $\textbf{coNP}$.
\end{note}

\begin{note}[Polynomial Hierarchy]
For $ i \geq$ 1, a language $L$ is in $\Sigma_{2}^{p}$ if
there exists a polynomial TM $M$ and a polynomial $q$ such that
$x \in L \iff \exists u_{1} \in \{0,1\}^{q(|x|)}
  \forall u_{2} \in \{0,1\}^{q(|x|)} \ldots
Q_{i}u_{i} \in \{0,1\}^{q(|x|)} M(x, u_{1}, \ldots, u_{i}) = 1 $,
where $Q_i$ denotes $\forall$ or $\exists$ depending on
whether $i$ is odd or even respectively.
The $\textit{polynomial hierarchy}$ is the set $\textbf{PH} = \cup_{i} \Sigma_{i}^{p}$.
\end{note}

\begin{note}
$\sum_{1}^{p} = \textbf{NP}. 
\Pi_{1}^{p} = \textbf{coNP}$. 
For every $i$, $\Sigma_{i}^{p} \subseteq \Pi_{i+1}^{p} \subseteq \Sigma_{i}^{p}.$ 
Thus, $\textbf{PH} = \cup_{i > 0} \Pi_{i}^{p}$.
\end{note}

\begin{note}[Collapse]
We believe that the polynomial hierarchy does not collapse.

\begin{enumerate}
	\item
		For every $i \geq 1$, if $\Sigma_{i}^{p} = \Pi_{i}^{p}$, then 
		$\textbf{PH} = \cup_{i} \Sigma_{i}^{p}$; that is, the polynomial hierarchy collapses
		to the $i^{th}$ level.
	\item
		If $\textbf{P} = \textbf{NP}$, then $\textbf{PH} = \textbf{P}$; that is, the polynomial
		hierarchy collapses to $\textbf{P}$.
\end{enumerate}
\end{note}

\begin{note}[Sketch for Collapse at $\PTIME = \NP$]
  Assume $\PTIME = \NP$ and induction on $i$.
  In the base case, we have that $\Sigma_{1}^p = \NP$ and
  $\Pi_{1}^p = \coNP$.
  Assume now that it holds to $i - 1$, and show
  that $\Sigma_{i}^p \subseteq P$.
  Because $\Pi_i ^p$ consists of the complements of languages in $\Sigma_i ^p$
  and $P$ is closed under complementation, it would also then follow that
  $\Pi_i ^p \subseteq P$.

  Let $L \in \Sigma_{i} ^p$, so by definition there is a TM $M$ and
  polynomial $q$ such that:
  \begin{align*}
    x \in L \iff \exists u_1 \in \{0, 1\}^{q(|x|)} \dots
    Q_i u_i \in \{0, 1\}^{q(|x|)} \st M(x, u_1, \dots, u_i) = 1
  \end{align*}

  Define a new $L^\prime$ as follows:
  \begin{align*}
    (x, u_1) \in L^\prime \iff
    \forall u_2 \in \{0, 1\}^{q(|x|)} \dots Q_i u_i \in \{0, 1\}^{q(|x|)}
    \st M(x, u_1, u_2, \dots, u_i) = 1
  \end{align*}
  Thus $L^\prime \in \Pi_{i - 1} ^p$ and by assumption, $L^\prime \in P$.
  Let $M^\prime$ be the TM that computes $L^\prime$, we plug $M^\prime$ back
  into the definition above and see that:
  \begin{align*}
    x \in L \iff \exists u_1 \in \{0, 1\}^{q(|x|)}
    \st M^\prime (x, u_1) = 1
  \end{align*}
  This means that $L \in \NP$, and hence under our assumption that
  $\PTIME = \NP$, we have $L \in \PTIME$.
\end{note}

\begin{note}
For every $i \in N$, both $\Sigma_{i}^{p}$ and $\Pi_{i}^{p}$ have
complete problems.
By contrast the polynomial hierarchy itself is believed not to have a
complete problem, as is shown by the following simple claim:

If there exists a language $L$ that is $\textbf{PH}$-complete,
then there exists an $i$ such that $\textbf{PH} = \Sigma_{i}^{p}$
(and hence the hierarchy collapses to its $i^{th}$ level.)
\end{note}

\begin{note}[Sketch for Collapse at the $i$th Level if $\PH$-completeness]
  Since $L \in \PH = \bigcup_i \sigma_i ^p$, there exists $i$ such that
  $L \in \Sigma_i ^p$.
  Since $L$ is $\PH$-complete, we can reduce every language of $\PH$ to $L$.
  But eveyr languae that is polynomial-time reducible to a language in
  $\Sigma_i ^p$ is itself in $\Sigma_i ^p$, and os we have shown that
  $\PH \subseteq \Sigma_i ^p$.
\end{note}

\begin{note}
$\textbf{PH} $is contained in PSPACE.
A simple corollary of the previous claim is that unless the
polynomial hierarchy collapses, $\textbf{PH} \neq \textbf{PSPACE}$.
Indeed, otherwise the $\PSPACE$-complete problem $\TQBF$ defined in
Section 4.2 would be PH-complete.
\end{note}

\begin{note}[Alternating Turing Machine]
Alternating TMs are a generalization of non-deterministic TMs.
Alternating TMs are similar to $\NDTM$s in the sense that they have two
transition functions between which they can choose in each step, but they
also have the additional feature that every internal state except
$q_{accept}$ and $q_{halt}$ is labeled with either $\exists$ or $\forall$.
Similar to the $\NDTM$, an $\ATM$ can evolve at every step in two possible
ways.
Recall that a $\NDTM$ accepts its input if there exists some sequence of
choices that leads it to the state $q_{accept}$.
In an $\ATM$, this existential quantifier over each choice is replaced
with the appropriate quantifier according to the labels at each state.
The name refers to the fact that the machine may atlernate between states
labeled $\exists$ and $\forall$.
\end{note}

\begin{note}[Alternating Time]
For every $T : \mathbb{N} \to \mathbb{N}$, we say that an atlernating
TM $M$ runs in $T(n)$-time if for every input $x \in \{0, 1\}^\ast$ and
for every possible sequence of transition function choices, $M$ halts after
at most $T(n)$ steps.
We say that a language $L \in \ATIME(T(n))$ if there is a constant $c$ and
a $c\cdot T(n)$-time $\ATM$ $M$ such that for every $x \in \{0, 1\}^\ast$,
$M$ accepts $x$ iff $x \in L$.
We define accepting as follows:

Recall that $G_{M, x}$ denote sthe directed acyclic configuration graph on
$M$ on input $x$, where there is an edge from a configuration $C$ to
configuration $C^\prime$ iff $C^\prime$ can be reached from $C$ by one
step of $M$'s transition function.
We label some of the vertices in this graph $\ACCEPT$ by applying
the following rules until they cannot be applied anymore:
\begin{itemize}
  \item
    The configuration $C_{accept}$ where the machine is in $q_{accept}$ is
    labeled $\ACCEPT$.

  \item
    If a configuration $C$ is in a state labeled $\exists$ and there is an
    edge from $C$ to a configuration $C^\prime$ labeled $\ACCEPT$, then we
    label $C^\prime$ as $\ACCEPT$.

  \item
    If a configuration $C$ is in a state labeled $\forall$ and both the
    configurations $C^\prime$, and $C^{\prime \prime}$ that are reachable
    from $C$ in one step are labeled $\ACCEPT$, then we label $C$ as $\ACCEPT$.
\end{itemize}

We say that $M$ accepts $x$ if at the end of this process the starting
configuration $C_{start}$ is labeled $\ACCEPT$.
\end{note}

\begin{note}
For every $i \in N$, we define $\Sigma_{i} \TIME(T(n))$
(resp. $\Pi_{i}\TIME(T(n))$) to be the set of languages accepted by
a $T(n)$-time ATM $M$ whose initial state is labeled "$\exists$"
(resp. "$\forall$") and on which every input and on every (directed) path
from the starting configuration in the configuration graph, $M$ can alternate
at most $i$ - 1 times from states with one label to states with the
other label.
\end{note}

\begin{note}
For every $i \in N$, $\Sigma_{i}^{p} = \cup_{c}\Sigma_{i}\TIME(n^c)$ and
$\Pi_{i}^{p} = \cup_{c} \Pi_{i} \TIME(n^c)$.
\end{note}

\begin{note}
Letting $\textbf{AP} = \cup_{c} \ATIME(n^c)$, we have
that $\textbf{AP} = \textbf{PSPACE}$.

Proof: $\textbf{PSPACE} \subseteq \textbf{AP}$ follows since $\TQBF$ is
trivially in $\textbf{AP}$ (just "guess" values for each existentially
quantified variable using an $\exists$ state and for universally quantified
variables using a $\forall$ state, and do a deterministic polynomial-time
computation at the end) and every $\textbf{PSPACE}$ language reduces
to $\TQBF$. To show that $\textbf{AP} \subseteq \textbf{PSPACE}$ we can use a
recursive procedure similar to the one used to show
that $\TQBF \in \textbf{PSPACE}$. 
\end{note}

\begin{note}
  $\APSPACE = \EXP$, where $\APSPACE$ is $\PSPACE$ equiped with $\ATM$s.
\end{note}

\begin{note}
  $\PTIME \subseteq \NP \subseteq \PSPACE \subseteq \EXP \subseteq \NEXP
  \subseteq \EXPSPACE$.
  By the polynomial hierarchy theorem, we also have:
  $\PTIME \subsetneq \EXP$, $\NP \subsetneq \NEXP$, and
  $\PSPACE \subsetneq \EXPSPACE$.
\end{note}

\begin{note}[Time/Space Tradeoff for SAT]
  For every two functions $S, T : \mathbb{N} \to \mathbb{N}$, define
  $\TISP(T(n), S(n))$ to be the set of languages decided by a TM $M$ that on
  every input $x$ takes at most $O(T(|x|))$ time, and uses at most
  $O(S(|x|))$ cells of its work tapes.
\end{note}

\begin{note}[Hierarchy by Oracles]
  For every $i \geq 2$, $\Sigma_i ^p = \NP^{\Sigma_{i - 1} \SAT}$, where the
  latter class denotes the set of languages decided by polynomial time
  $\NDTM$s with access to the oracle $\Sigma_{i - 1} \SAT$.
\end{note}

%%% Not formatted / Incomplete %%%
%\begin{note}[Time/Space tradeoff for SAT]
%For every two functions S, T : N ? N, define TISP(T (n), S(n)) to be the set of languages decided by a TM M that on every input x takes at most O(T (|x|)) steps and uses at most O(S(|x|)) cells of its read/write tapes. Then, SAT?? TISP(n1.1, n0.1).
%\end{note}

%\begin{note}[Proof of Time/Space tradeoff for SAT]

%\end{note}

%\begin{note} [Oracle machines]
%For every i ? 2, ?pi = NP?i?1SAT, where the latter class denotes the set of languages decided by polynomial-time NDTMs with access to the oracle ?i?1SAT.
%\end{note}

%\begin{note} [Proof of SAT orcale PNDTMs]
%\end{note}











