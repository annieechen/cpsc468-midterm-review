\subsection*{Chapter 5: The Polynomial Hierarchy and Alterations}
\begin{note}[The class $\Sigma_{2}^{p}$]
The class $\Sigma_{2}^{p}$ is the set of all languages for which there exists a polynomial TM $M$
and a polynomial $q$ such that
$x \in L \iff \exists u \in \{0,1\}^{q(|x|)} \forall v \in \{0,1\}^{q(|x|)} M(x, u, v) = 1 $

Note that $\Sigma_{2}^{p}$ contains the classes $\textbf{NP}$ and $\textbf{coNP}$.
\end{note}

\begin{note}[Polynomial Hierarchy]
For $ i \geq$ 1, a language $L$ is in $\Sigma_{2}^{p}$ if there exists a polynomial TM $M$
and a polynomial $q$ such that
$x \in L \iff \exists u_{1} \in \{0,1\}^{q(|x|)} \forall u_{2} \in \{0,1\}^{q(|x|)} \ldots
Q_{i}u_{i} \in \{0,1\}^{q(|x|)} M(x, u_{1}, \ldots, u_{i}) = 1 $,
where $Q_i$ denotes $\forall$ or $\exists$ depending on whether $i$ is odd or even respectively.
The $\textit{polynomial hierarchy}$ is the set $\textbf{PH} = \cup_{i} \Sigma_{i}^{p}$.
\end{note}

\begin{note}
$\sum_{1}^{p} = \textbf{NP}. 
\Pi_{1}^{p} = \textbf{coNP}$. 
For every $i$, $\Sigma_{i}^{p} \subseteq \Pi_{i+1}^{p} \subseteq \Sigma_{i}^{p}.$ 
Thus, $\textbf{PH} = \cup_{i > 0} \Pi_{i}^{p}$.
\end{note}

\begin{note}[Collapse]
We believe that the polynomial hierarchy does not collapse.

\begin{enumerate}
	\item
		For every $i \geq 1$, if $\Sigma_{i}^{p} = \Pi_{i}^{p}$, then 
		$\textbf{PH} = \cup_{i} \Sigma_{i}^{p}$; that is, the polynomial hierarchy collapses
		to the $i^{th}$ level.
	\item
		If $\textbf{P} = \textbf{NP}$, then $\textbf{PH} = \textbf{P}$; that is, the polynomial
		hierarchy collapses to $\textbf{P}$.
\end{enumerate}
Prove 2 by induction on $i$ that $\Sigma_{i}^{p}, \Pi_{i}^{p} \subseteq \textbf{P}$.
\end{note}

\begin{note}
For every $i \in N$, both $\Sigma_{i}^{p}$ and $\Pi_{i}^{p}$ have complete problems. By contrast the polynomial hierarchy itself is believed not to have a complete problem, as is shown by the following simple claim:

If there exists a language $L$ that is $\textbf{PH}$-complete, then there exists an $i$ such that $\textbf{PH} = \Sigma_{i}^{p}$ (and hence the hierarchy collapses to its $i^{th}$ level.)
\end{note}

\begin{note}
$\textbf{PH} $is contained in PSPACE. A simple corollary of the previous claim is that unless the polynomial hierarchy collapses, $\textbf{PH} \neq \textbf{PSPACE}$. Indeed, otherwise the $\PSPACE$-complete problem $\TQBF$ defined in Section 4.2 would be PH-complete.
\end{note}

\begin{note}[Alternating Time]
For every $T : N \to N$, we say that an alternating TM $M$ runs in $T(n)$-time if for every input $x \in \{0, 1\}^{\ast}$ and for every possible sequence of transition function choices, $M$ halts after at most $T(|x|)$ steps.

We say that a language $L$ is in $\ATIME(T(n))$ if there is a constant $c$ and a $cT(n)$-time ATM $M$ such that for every $x \in \{0, 1\}^{\ast}$, $M$ accepts $x \iff x \in L$.
\end{note}

\begin{note}
For every $i \in N$, we define $\Sigma_{i} \TIME(T(n))$ (resp. $\Pi_{i}\TIME(T(n))$) to be the set of languages accepted by a $T(n)$-time ATM $M$ whose initial state is labeled "$\exists$" (resp. "$\forall$") and on which every input and on every (directed) path from the starting configuration in the configuration graph, $M$ can alternate at most $i$ - 1 times from states with one label to states with the other label.
\end{note}

\begin{note}
For every $i \in N$, $\Sigma_{i}^{p} = \cup_{c}\Sigma_{i}\TIME(n^c)$ and $\Pi_{i}^{p} = \cup_{c} \Pi_{i} \TIME(n^c)$.
\end{note}

\begin{note}
Letting $\textbf{AP} = \cup_{c} \ATIME(n^c)$, we have that $\textbf{AP} = \textbf{PSPACE}$.

Proof: $\textbf{PSPACE} \subseteq \textbf{AP}$ follows since $\TQBF$ is trivially in $\textbf{AP}$ (just "guess" values for each existentially quantified variable using an $\exists$ state and for universally quantified variables using a $\forall$ state, and do a deterministic polynomial-time computation at the end) and every $\textbf{PSPACE}$ language reduces to $\TQBF$. To show that $\textbf{AP} \in \textbf{PSPACE}$ we can use a recursive procedure similar to the one used to show that $\TQBF \in \textbf{PSPACE}$. 
\end{note}


%%% Not formatted / Incomplete %%%
%\begin{note}[Time/Space tradeoff for SAT]
%For every two functions S, T : N ? N, define TISP(T (n), S(n)) to be the set of languages decided by a TM M that on every input x takes at most O(T (|x|)) steps and uses at most O(S(|x|)) cells of its read/write tapes. Then, SAT?? TISP(n1.1, n0.1).
%\end{note}

%\begin{note}[Proof of Time/Space tradeoff for SAT]

%\end{note}

%\begin{note} [Oracle machines]
%For every i ? 2, ?pi = NP?i?1SAT, where the latter class denotes the set of languages decided by polynomial-time NDTMs with access to the oracle ?i?1SAT.
%\end{note}

%\begin{note} [Proof of SAT orcale PNDTMs]
%\end{note}











